\documentclass[a4paper]{article}

\usepackage[english]{babel}
\usepackage[utf8x]{inputenc}
\usepackage{amsmath}
\usepackage{graphicx}
\usepackage[colorinlistoftodos]{todonotes}
\usepackage[margin=1.2in]{geometry}
\usepackage{amssymb}

\title{Skupina 8-Metric independence number}
\author{Živa Kocijan, Lucija Koprivec}
\date{14.12.2023}

\begin{document}
\maketitle

\section{Naloga}
Najine naloge so:
\begin{itemize}
  \item Among trees $T$ on $n$ vertices, find trees with maximum/minimum $mi(T)$.
  \item Find trees for which $dim(T) = mi(T)$. Here, $dim(T)$ is the classical vertex metric dimension.
  \item Find trees for which $dim(T)\text{--} mi(T)$ is maximum/minimum.
  \item Determine $mi(G)$ of a grid graph $G = P_{k} \square P_{t}$
  \item Determine $mi(G)$ of a hypercube $G = Q_{d}$
\end{itemize}


\section{Osnovni pojmi}
Naj bo $G$ graf. Točka $x$ grafa $G$ razreši dve točki $u$ in $v$ grafa $G$, če je
razdalja med $x$ in $u$ različna od razdalje med $x$ in $v$.\\
Niz $S$ oglišč $G$ je razrešujoča množica za $G$, če sta vsaki dve različni točki
iz $G$ razrešeni z nekim ogliščem iz $S$. Najmanjša moč
razrešujoče množice za $G$ se imenuje metrična dimenzija $G$ in jo označimo z $dim(G)$. \\
Iskanje metrične dimezije grafa naj bi bil NP-popoln.\\
Naprej, naj bo $V_{p}=\{\{x,y\}\, x,y \in V(G)\}$ in neka njena podmonžica $P$ je neodvisen razrešujoč nabor parov, če ni noben par iz P razrešen z istim ogliščem. Pri tem pa z $mi(G)$ označimo metrično neodvisno število grafa $G$, ki je moč največje množice P. \\
Število $mi(G)$ je možno najti s pomočjo celoštevilskega linearnega programa.

\section{Vrste grafov}
\textbf{Drevo} je neusmerjen graf, v katerem sta katerikoli dve točki povezani z natanko eno potjo.\\
\textbf{Hiperkocka} $Q_{d}$ je graf, ki ima $2^{n}$ vozlišč in vsako vozlišče je stopnje $n$. \\
\textbf{Produkt grafov} $G\square H$, je nov graf, ki ima število oglišč enako $V(G)*V(H)$


\section{NP-popolnost}
V računalništvu je NP-popolnost ali NP-trdnost problema merilo težavnosti reševanja tega problema. Problem je NP-popoln, če ga je mogoče rešiti s polinomskim časovnim algoritmom in če je tudi NP-težek
V angleščini je ime "NP-complete" okrajšava za "nondeterministic polynomial-time complete". 

\section{CLP}
Celoštevisli linearni program ali krajše CLP je matematični optimizacijski 
problem, pri katerem je cilj maksimizirati oziroma minimizirati linearno 
funkcijo celoštevilskih spremenljivk ob upoštevanju določenega nabora 
linearnih enakosti in neenakosti.
Drugače povedano, CLP je vrsta matematičnega programiranja, kjer so 
celoštevilske vrednosti zahtevane za odločitvene spremenljivke.

\section{Primer CLP-ja pri iskanju metrične dimenzije grafa}
Naj bo G graf, potem definiramo 
\begin{equation} 
  a_{p,v} =
    \begin{cases}
      1& \text{če je $p \in V_{p}$ razrešen z $v \in V(G)$}\\
      0& \text{sicer}
    \end{cases}       
\end{equation}
\begin{align}
max &\sum_{p \in V_{p}} y_{p} &\\ \nonumber
    s.t. \sum_{p \in V_{p}} a_{p,v}y_{p} & \leq 1 &\text{za vsak} v \in V(G), \\ \nonumber
    y_{p} & \in \{0,1\} &\text{za vsak} p \in V_{p}, \nonumber
\end{align}
kjer je $y_{p}=1$, če in samo če $p \in P$.

\section{Primerjava vrednosti}
Za vse grafe, ki jih označimo z $G$, velja neeakost:
$$mi(G) \leq dim(G)$$
Problem, če za dani graf $G$ v zgornji neenačbi velja enakost, je prav tako NP-popoln. \\
Sledi, da je $0 \leq dim(G)-mi(G) \leq dim(G)$.\\

\end{document}
