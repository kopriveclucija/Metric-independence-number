\documentclass[a4paper]{article}

\usepackage[english]{babel}
\usepackage[utf8x]{inputenc}
\usepackage{amsmath}
\usepackage{graphicx}
\usepackage[colorinlistoftodos]{todonotes}
\usepackage[margin=1.2in]{geometry}
\usepackage{amssymb}
\usepackage{multirow}
\usepackage{lipsum}
\usepackage{caption}
\captionsetup[table]{name=Tabela}


\title{Skupina 8 - Metric independence number}
\author{Živa Kocijan, Lucija Koprivec}
\date{19.1.2024}

\begin{document}
\maketitle

\section{Naloga}
Najine naloge so:
\begin{itemize}
  \item Among trees $T$ on $n$ vertices, find trees with maximum/minimum $mi(T)$.
  \item Find trees for which $dim(T) = mi(T)$. Here, $dim(T)$ is the classical vertex metric dimension.
  \item Find trees for which $dim(T)\text{--} mi(T)$ is maximum/minimum.
  \item Determine $mi(G)$ of a grid graph $G = P_{k} \square P_{t}$
  \item Determine $mi(G)$ of a hypercube $G = Q_{d}$
\end{itemize}

\section{Osnovne oznake}
\begin{itemize}
  \item $mi(G)$, bo označevalo metrično neodvisno število grafa $G$.
  \item $dim(G)$, bo označevalo metrično dimenzijo grafa $G$.
  \item $T$, bo označevalo graf, ki je drevo. Torej povezan graf brez ciklov.
  \item $P_{k}$, graf ki je pot, na $k$ vozliščih.
  \item $G = K \square L$, graf $G$, ki je podukt grafa $K$, $L$. Tu je množica točk $V(K \square L)$ definirana kot, kartezični  produkt točk grafov $V(K) \times V(L)$. Množica povezav $E(K \square L)$ pa jemnožica vseh parov točk $(u,v)(x,y)$, za katerega velja bodisi $u=x$ in $(v,y) \in E(L)$ bodisi $(u,x) \in E(G)$ in $v=y$.
  \item $Q_{d}$, bo označeval graf, ki je tako imenovana hiperkocka. To je graf, ki ima $2^{n}$ vozlišč, $n2^{n-1}$ povezav in je $n-regularen$.
\end{itemize}

\section{Jasen opis problema}
Naj bo $G$ graf. Točka $x$ grafa $G$ razreši dve točki $u$ in $v$ grafa $G$, če je $d(x,u) \neq d(x,v)$\\
Niz $S$ oglišč $G$ je razrešujoča množica za $G$, če sta vsaki dve različni točki
iz $G$ razrešeni z nekim ogliščem iz $S$. Najmanjša moč
razrešujoče množice za $G$ se imenuje metrična dimenzija $G$ in jo označimo z $dim(G)$. \\
Naprej, naj bo $V_{p}=\{\{x,y\}; \; x,y \in V(G)\}$ in neka njena podmonžica $P$ je neodvisen razrešujoč nabor parov, če ni noben par iz P razrešen z istim ogliščem. Pri tem pa z $mi(G)$ označimo metrično neodvisno število grafa $G$, ki je moč največje množice P. \\
Pri tem lahko za množico $V_{p}$ vzamemo tudi množico vseh  parov različnih točk. Obe definiciji sta ekvivalentni, le da v prvem primeru vsebuje množica $V_{p}$, $n + {n\choose2}$ elementov, v drugem primeru pa ${n\choose2}$, kjer je $n$ število ogljišč grafa.\\
V obeh primerih, sva vzeli neurejene pare vozlišč.\\
Če ne piše drugače, sva delali z množico vseh parov različnih vozlišč.\\
Določanje števila $mi$, je mogoče prevesti na linearen program.\\
S spreminjanjem lastnosti grafov, se tako spreminjajo tudi vrednosti $mi$ in $dim$. Za razične skupne lastnosti grafov (drevo, hiperkocka, kartezični produkti poti...) sva iskali vrednosti in ujemanja, glede na obliko grafa, iskanih vrednosti.


\section{Glavne ideje programa}
\subsection{Metrično neodvisno število}
Glavna problem iskanja števila $mi$, je napisati linearni program v Sage-u. Tako dobimo, kot rešitev linearnega programa, metrično neodvisno število grafa, ki nas zanima.\\
Za ta namen sva definirali "matriko" $A$, ki ima vrednosti $0$ ali $1$. Zanjo velja $A \in \mathbb{R}^{n \times m}$, kjer je $n$ število vozlišč grafa, $m$ pa je število neurejenih parov grafa, torej $m={n\choose2}$.
\begin{equation} 
  A_{i,j} =
    \begin{cases}
      1& \text{če je $i$-to vozlišče razrešeno z $j$-tim parom vozlišč}\\
      0& \text{sicer}
    \end{cases}       
\end{equation}
Nato sva iskali maksimum funkcije $$\sum_{j \in V_{p}} y_{j}, $$ pri čemer morajo veljati omejitve $$\sum_{j \in V_{p}} A_{j,i}y_{j} \leq 1,  $$ za vsako vozlišče $v$ našega grafa, torej imava v linearnem programu $n$ omejitev. Pri tem je vrednost nove spremenljivke $y_j=0\;
 \text{ali}\; 1$; $1$ v primeru, da par $j$ pripada podmnožici $P$, množice $V_p$.\\
 Iskali sva torej največjo moč množice $P$, za katero noben par točk, ni razrešen z istim vozliščem. 
 
\subsection{Metrična dimenzija}
Pri iskanju metrične dimenzije, sva s pomočjo funkcije \texttt{subsets}, ustvarili množico neurejenih parov različnih vozlišč. Nato sva za vsak par poiskali vozlišča grafa, ki par razrešijo.\\
Za vsako podmnožico vozlišč grafa, sva preverili, ali razreši vse pare vozlišč ali ne, in izmed podmnožic, ki razrešijo vse pare vozlišč, izračunali moč najmanjše.\\
To je hkrati vrednost metrične dimenzije grafa.

\section{Zbiranje podatkov}
S pomočjo okolja Sage, sva napisali dve funkciji.\\
Ena je spisana kot linearni program (opisan zgoraj) za izračunavanje neodvisnega metričnega števila za poljuben dani graf. \\
Drugo sva uporabili za izračunavanje metrične dimenzije poljubnega danega grafa.\\
Nato sva s pomožnimi funkcijami generirali razične vrste grafov in na njih računali iskane vrednosti.\\
Pri prvih treh primerih sva se ukvarjali z drevesi. Tako sva pri različnih številih vozlišč drevesa iskale "obliko" grafa, kjer je $mi$ čim večji oziroma čim manjši in le to vrednost "primerjali" z $dim$.\\
V četrtem primeru sva generirali grafe kartezičnih produktov dveh poti. V petem sva ugotavljali $dim$ za hiperkocke in sva tako generirali hiperkocke različnih dimenzij.\\
Tako sva iz generiranih primerov na "manjših" grafih in dobljenih vrednosti $dim$ ter $mi$, poizkušale sklepati, kakšne so vrednosti na "vseh" grafih, katere zahteva najina naloga.



\section{Rezultati}


\subsection{Prva alineja}
Pri drevesih v teroriji grafov na $n$ vozliščih je možno opaziti, $$\text{število povezav}  = n-1. $$ \\
Pri iskanju minimalnega in maksimalnega $mi(G)$, se lahko sprva omejimo z $0 \leq mi(G) \leq dim(G)$. Vemo pa da vsak par vozlišč v povezanem grafu razrešita vsaj točki para. Primer: $d(u,v) \neq d(v,v)$ in $d(u,u) \neq d(v,u)$. Tako se lahko pri drevesih $T$ omejimo še nekoliko bolj in velja $$mi(T) \leq \lfloor \frac{n}{2} \rfloor, $$ kjer je $n$ število vozlišč drevesa. \\
Izkaže se, da je za vsa možna drevesa, s številom vozlišč od dva do štiri, vrednost največjega in najmanjšega $mi$ enaka 1. Pri višjem številu vozlišč se pojavi vzorec; minimalno vrednost $mi$ imajo drevesa oblike poti. Za drevesa z lihim številom vozlišč imajo največjo vrednost $mi$ drevesa oblike zvezde ("centralno" središče, povezano z vsemi ostalimi), za drevesa s sodim številom vozlišč pa imajo največjo vrednost $mi$ drevesa oblike dvojne zvezde (dve "centralni" središči, vsako povezano s polovico preostalih vozlišč).
\begin{figure}[h!]
    \centering
    \includegraphics[width=0.5\linewidth]{9vozlisc.png}
    \caption{Primer dreves z 9 vozlišči. Prvi oblike poti (najmanjši $mi$), drugi oblike zvezde (največji $mi$).}
    \label{fig:enter-label}
\end{figure}
\begin{figure}[h!]
    \centering
    \includegraphics[width=0.5\linewidth]{10vozlisc.png}
    \caption{Primer dreves z 10 vozlišči. Prvi oblike poti (najmanjši $mi$), drugi oblike dvojne zvezde (največji $mi$).}
    \label{fig:enter-label}
\end{figure}

\subsection{Druga alineja}
V tem primeru sva iskali drevesa, za katere sta vrednosti $dim$ in $mi$ enaki. Taka so samo drevesa z dvema vozliščema, povezanima z eno povezavo. Vsa drevesa z več vozlišči imajo ti dve vrednosti različni
\subsection{Tretja alineja}
Kot omenjeno že v prejšnji točki, imajo najmanjšo oz. ničelno razliko med vrednostima $dim$ in $mi$ drevesa z dvema vozliščema. Za taka drevesa je tudi največja razlika ničelna. Pri drevesih s  tremi ali štirimi vozlišči sta maksimalna in minimalna razlika med $dim$ in $mi$ enaki in sicer je razlika za drevesa s tremi vozlišči enaka 1 in za drevesa s štirimi vozlišči enaka 2. Pri drevesih z več vozlišči se pojavi vzorec; najmanjša razlika med $dim$ in $mi$ se pojavlja pri drevesih oblike zvezde (liho število vozlišč) oziroma dvojne zvezde (sodo število vozlišč), medtem ko je največja razlika pri drevesih oblike poti. Opazimo, da je ta vzorec obraten v primerjavi z najmanjšo/največjo vrednostjo $mi$, ki je opisana pri prvi alineji.
\subsection{Četrta alineja}
Kartezični produkti grafov imajo lastnost asiociativnosti in komutativnosti. Tako je graf $P_n \square P_m$  izomorfen grafu $P_m \square P_n$.\\
V programu sva tako vzeli produkte grafov, ki so predstavljeni v spodnji tabeli. Zaradi komutativnosti sva pri iskanju vrednosti $mi$, delale z grafi oblike $P_m \square P_n$, kjer je $n \leq m$.\\
Za primerjavo sva vrednost $mi$ računali na dva načina.\\
\textbf{Prvi način}:\\
Za množico $V_p$ sva vzeli, kot prej, neurejene pare različnih vozlušč.\\
\textbf{Drugi način}:\\
Za množico $V_p$ sva vzeli, poleg neurejenih parov različih vozlišč, tudi pare oblike $\{\{x,x\};\; x \in V(G)\}$.\\
Graf $P_1 \square P_1$ nima nobenega para vozlišč, zato je $mi=0.$ Produkti grafov oblike $P_1 \square P_n$, kjer je $n>1$ imajo $mi=1$, vsi drugi pa $2$.
Če primerjamo vrednost $mi$-ja na prvi in na drugi način, lahko opazimo, da se razlikuje za število vozlišč danega grafa.
\begin{table}[h!]
\centering
\begin{tabular}{||c c c c||} 
 \hline
 dolžina krajše poti & dolžina daljše poti & mi na prvi način & mi na drugi način \\ [0.5ex] 
 \hline\hline
 1 & 1 & 0 & 1 \\ 
 1 & 2 & 1 & 3 \\
 1 & 3 & 1 & 4 \\
 1 & 4 & 1 & 5 \\
 1 & 5 & 1 & 6 \\
 1 & 6 & 1 & 7 \\ 
 \hline 
 2 & 2 & 2 & 6  \\
 2 & 3 & 2 & 8  \\
 2 & 4 & 2 & 10  \\
 2 & 5 & 2 & 12  \\
 2 & 6 & 2 & 14  \\ 
 \hline
 3 & 3 & 2 & 11 \\
 3 & 4 & 2 & 14 \\
 3 & 5 & 2 & 17 \\
 3 & 6 & 2 & 21 \\ 
 \hline
 4 & 4 & 2 & 18 \\
 4 & 5 & 2 & 22 \\
 \hline
 5 & 5 & 2 & 27 \\
 5 & 6 & 2 & 32 \\ 
 \hline
 6 & 6 & 2 & 38\\
 \hline
 7 & 7 & 2 & 51 \\
 \hline
 8 & 8 & 2 & 66 \\
 \hline
\end{tabular}
\caption{Tabela za produkt poti.}
\label{table:1}
\end{table}


\subsection{Peta alineja}
Podobno kot pri četrti alineji sva primerjali dve vrednosti.\\
Hiperkocka $Q_1$ ima 2 vozlišča, torej samo en par različnih vozlišč in je $mi=1$, druge hiperkocke pa imajo $mi=2$.
Če primerjamo vrednosti na drug način je ponovno možno opziti, da se $mi$-ja razlikujera za število vozlišč.
\begin{table}[t!]
\centering
\begin{tabular}{||c  c c||} 
 \hline
 hiperkocka & mi na prvi način & mi na drugi način \\ [0.5ex] 
 \hline\hline
 $Q_{1}$ & 1 & 3 \\ 
 $Q_{2}$ & 2 & 6 \\
 $Q_{3}$ & 2 & 10 \\
 $Q_{4}$ & 2 & 18 \\
 $Q_{5}$ & 2 & 34 \\
 $Q_{6}$ & 2 & 66 \\ 
 \hline
\end{tabular}
\caption{Tabela za hiperkocke.}
\label{table:2}
\end{table}
\end{document}
